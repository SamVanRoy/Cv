% a mashup of hipstercv, friggeri and twenty cv
% https://www.latextemplates.com/template/twenty-seconds-resumecv
% https://www.latextemplates.com/template/friggeri-resume-cv

\documentclass[lightTheme]{cv}
% available options are: darkTheme, lightTheme, pastel, allblack, grey, verylight, withoutsidebar
% withoutsidebar
\usepackage[utf8]{inputenc}
\usepackage[default]{raleway}
\usepackage[margin=1cm, a4paper]{geometry}
\usepackage{supertabular}

%------------------------------------------------------------------ Variablen

\newlength{\rightcolwidth}
\newlength{\leftcolwidth}
\setlength{\leftcolwidth}{0.23\textwidth}
\setlength{\rightcolwidth}{0.75\textwidth}

%------------------------------------------------------------------
\title{Cv Sam NL}
\author{Sam Van Roy}
\date{May 2022}

\pagestyle{empty}
\begin{document}


\thispagestyle{empty}
%-------------------------------------------------------------

\section*{Start}

\simpleheader{headercolour}{Sam}{Van Roy}{Java lead developer}{white}{OCA_Badge.png}{OCP_Badge.png}



%------------------------------------------------

% this has to be here so the paracols starts..
\subsection*{}
\vspace{4em}

\setlength{\columnsep}{1.5cm}
\columnratio{0.23}[0.75]
\begin{paracol}{2}
\hbadness5000
%\backgroundcolor{c[1]}[rgb]{1,1,0.8} % cream yellow for column-1 %\backgroundcolor{g}[rgb]{0.8,1,1} % \backgroundcolor{l}[rgb]{0,0,0.7} % dark blue for left margin

\paracolbackgroundoptions

% 0.9,0.9,0.9 -- 0.8,0.8,0.8


\footnotesize
{\setasidefontcolour
\flushleft
\begin{center}
    \roundpic{jack.jpg}
\end{center}

\bg{cvgreen}{white}{Persoonlijke info} \\[0.5em]

\icon{\faCalendar}{cvgreen}{} 2 februari 1997 \\[0.5em]

\icon{\faMapMarker}{cvgreen}{} 9280 Lebbeke \break
(Regio Oost-Vlaanderen) \\[0.5em]

\icon{\faMapMarker}{cvgreen}{} 1730 Asse \break
(Regio Vlaams-Brabant) \\[0.5em]

\icon{\faAt}{cvgreen}{} saxo4sam@gmail.com \\[0.5em]

\icon{\faLinkedinSquare}{cvgreen}{} www.linkedin.com/in/ \break
sam-van-roy-457528140

\bigskip

\bg{cvgreen}{white}{Opleiding} \\[0.5em]
Toegepaste informatica \\[0.5em]
Erasmushogeschool Brussel \\[0.5em]
2015 - 2018 \\[0.4em]

\bigskip

\bg{cvgreen}{white}{Certificaten} \\[0.5em]
\hspace*{-0.8em}\begin{tabular}{l l}
2021 & 1Z0-809: OCP, Java SE 8 \\[0.5em]
2019 & 1Z0-808: OCA, Java SE 8
\end{tabular}

\bigskip

\bg{cvgreen}{white}{Talen} \\[0.5em]
\hspace{-0.8em}\begin{tabular}{l | l}
Nederlands & {\phantom{x}\footnotesize moedertaal} \\[0.5em]
Engels & \pictofraction{\faCircle}{cvgreen}{3}{black!30}{1}{\tiny} \\[0.5em]
Frans & \pictofraction{\faCircle}{cvgreen}{2}{black!30}{2}{\tiny}
\end{tabular}

\bigskip

}
%-----------------------------------------------------------
\switchcolumn

\small

\section*{Eerste kennismaking}

Sinds mijn jonge jaren wist ik al wat ik wou worden. Zovele leerrijke jaren later heb ik het geluk om één van de personen te zijn die van zijn interesse ook effectief zijn job heeft kunnen maken. Mijn \textbf{interesse} in informatica vloeit voort uit de constante nieuwe evoluties en nieuwe \textbf{uitdagingen} die informatica steeds weer kent. Zo experimenteer ik ook al graag eens met \textbf{nieuwe technologieën}. Een mooi voorbeeld hiervan is mijn bachelorproef met de Microsoft HoloLens. \\

Een belangrijk kenmerk van mijzelf is dat ik het leven steeds \textbf{positief} probeer aan te pakken en met een \textbf{“het gaat lukken”-mentaliteit}. Daarnaast kan ik mij enorm vastbijten in zaken, zeker wanneer deze niet verlopen zoals het zou kunnen. Zelf vind ik naast \textbf{eerlijkheid} het zeer belangrijk op het werk om een goede band te creëren met de collega’s. Het is namelijk belangrijk om het opleveren van werk te kunnen doen in de nodige \textbf{toffe sfeer}.

\section*{Werkervaring}

\begin{supertabular}{r| p{0.5\textwidth} c}
    \cvevent{\cvperiod[t]{10/2018\\-\\10/2018}}{Full stack developer}{Inetum-Realdolmen (intern project)}{
        \cvproject{ToGetAir}{Multifunctionele reisagentschapstoepassing voor het:
            \begin{itemize}
                \item zoeken en boeken van bestemmingen (klanten).
                \item beheren van luchtvaartmaatschappijen, vluchtroutes, kortingen en prijzen (reismakelaars).
                \item beheren van winstmarges en promoties (werknemers).
            \end{itemize}
            Dit project werd ontwikkeld als onderdeel van het AcADDemICT opleidingstraject en zette in op planning en communicatie met de teamleden en de productowner. }
        \cvproject{Takenpakket en verwezenlijkingen}{
            \begin{itemize}
                \item Bedenken en analyseren opdracht en architectuur met team.
                \item Implementeren van de nodige logica en schermen om bepaalde entiteiten (vliegmaatschappijen, luchthavens,…) aan te maken, bekijken en aanpassen. (CRUD)
                \item Globalexceptionhandler om alle mogelijke fouten in het programma op te vangen.
                \item Routing van pagina’s.
            \end{itemize}}
        \cvproject{Technologieën en frameworks}{Java SE8, JSF, PrimeFaces, JUnit, Mockito, Hibernate / JPA, Ajax, JQuery, CSS, HTML, Materialise}
        \cvproject{Tools}{Git (Gitlab/GitKraken), Maven, IntelliJ IDEA, WildFly, MySQL, Jira, Postman}
        \cvproject{Methodologieën}{Agile (Scrum), CI, Kanban}
    }{Inetum-Realdolmen.jpg} \\
    \cvevent{\cvperiod[t]{02/2018\\-\\05/2018}}{Full stack developer}{Stage bij Telenet i.o.v. Ausy \color{cvred}}{
        \cvproject{Eos Tv-shop}{Voor de nieuwe digicorder van Telenet werd een nieuwe Tv-shop ontwikkeld waarbij klanten niet enkel meer online producten konden bestellen, maar nu ook via je digicorder.}
        \cvproject{Proof of concept Jenkins 2}{Proof of concept voor de overstap van Jenkins 1 naar Jenkins 2 voor verschillende teams  binnen Telenet.}
        \cvproject{Takenpakket en verwezenlijkingen}{
            \begin{itemize}
                \item Onderzoek naar mogelijkheden van Jenkins 2 voor volledige afdeling E-Development van Telenet.
                \item Webapplicatie in team voor Tv-shop in Angular 5. Deze werd volledig van scratch opgebouwd.
                \item De uitdaging bestond hier om de applicatie en de animaties zoveel mogelijk te optimaliseren door de beperkte resources van een digicorder.
            \end{itemize}}
        \cvproject{Technologieën en methodieken}{Angular 5, Typescript, Karma, Intellij IDEA, Git, JIRA, Jenkins 2, Agile}
    }{telenet.png}
\end{supertabular}
\vspace{3em}


\section*{Short Resumé}

\begin{tabular}{r| p{0.5\textwidth} c}
    \cvevent{2018--2021}{Captain of the Black Pearl}{Lead}{East Indies \color{cvred}}{Finally got the goddamn ship back.\lorem\lorem\lorem}{disney.png} \\
    \cvevent{2016--2017}{Captain of the Black Pearl}{Lead}{Tortuga \color{cvred}}{Found a secret treasure, lost the ship. \lorem\lorem}{medal.jpeg}
\end{tabular}
\vspace{3em}

\begin{minipage}[t]{0.35\textwidth}
\section*{Degrees}
\begin{tabular}{r p{0.6\textwidth} c}
    \cvdegree{1710}{Captain}{Certified}{Tortuga Uni \color{headerblue}}{}{disney.png} \\
    \cvdegree{1715}{Bucaneering}{M.A.}{London \color{headerblue}}{}{medal.jpeg} \\
    \cvdegree{1720}{Bucaneering}{B.A.}{London \color{headerblue}}{}{medal.jpeg} \\
\end{tabular}
\end{minipage}\hfill
\begin{minipage}[t]{0.3\textwidth}
\section*{Programming}
\begin{tabular}{r @{\hspace{0.5em}}l}
     \bg{skilllabelcolour}{iconcolour}{html, css} &  \barrule{0.4}{0.5em}{cvpurple}\\
     \bg{skilllabelcolour}{iconcolour}{\LaTeX} & \barrule{0.55}{0.5em}{cvgreen} \\
     \bg{skilllabelcolour}{iconcolour}{python} & \barrule{0.5}{0.5em}{cvpurple} \\
     \bg{skilllabelcolour}{iconcolour}{R} & \barrule{0.25}{0.5em}{cvpurple} \\
     \bg{skilllabelcolour}{iconcolour}{javascript} & \barrule{0.1}{0.5em}{cvpurple} \\
\end{tabular}
\end{minipage}

\section*{Curriculum}
\begin{tabular}{r| p{0.5\textwidth} c}
    \cvevent{2018--2021}{Captain of the Black Pearl}{Lead}{East Indies \color{cvred}}{Finally got the goddamn ship back. \lorem}{disney.png} \\
    \cvevent{2019}{Freelance Pirate}{Bucaneering}{Tortuga \color{cvred}}{This and that. The usual, aye?  \lorem}{medal.jpeg} \\
\end{tabular}
\vspace{3em}

\begin{minipage}[t]{0.3\textwidth}
\section*{Certificates \& Grants}
\begin{tabular}{>{\footnotesize\bfseries}r >{\footnotesize}p{0.55\textwidth}}
    1708 & Captain's Certificates \\
    1710 & Travel grant \\
    1715--1716 & Grant from the Pirate's Company
\end{tabular}
\bigskip

\section*{Languages}
\begin{tabular}{l | ll}
\textbf{English} & C2 & {\phantom{x}\footnotesize mother tongue} \\
\textbf{French} & C2 & \pictofraction{\faCircle}{cvgreen}{3}{black!30}{1}{\tiny} \\
\textbf{Spanish} & C2 & \pictofraction{\faCircle}{cvgreen}{1}{black!30}{3}{\tiny} \\
\textbf{Italian} & C2 & \pictofraction{\faCircle}{cvgreen}{3}{black!30}{1}{\tiny}
\end{tabular}
\bigskip

\end{minipage}\hfill
\begin{minipage}[t]{0.3\textwidth}
\section*{Publications}
\begin{tabular}{>{\footnotesize\bfseries}r >{\footnotesize}p{0.7\textwidth}}
    1729 & \emph{How I almost got killed by Lady Swan}, Tortuga Printing Press. \\
    1720 & ``Privateering for Beginners'', in: \emph{The Pragmatic Pirate} (1/1720).
\end{tabular}
\bigskip

\section*{Talks}
\begin{tabular}{>{\footnotesize\bfseries}r >{\footnotesize}p{0.6\textwidth}}
    Nov. 1726 & ``How I lost my ship (\& and how to get it back)'', at: \emph{Annual Pirate's Conference} in Tortuga, Nov. 1726.
\end{tabular}
\end{minipage}






\vfill{} % Whitespace before final footer

%----------------------------------------------------------------------------------------
%	FINAL FOOTER
%----------------------------------------------------------------------------------------
\setlength{\parindent}{0pt}
\begin{minipage}[t]{\rightcolwidth}
\begin{center}\fontfamily{\sfdefault}\selectfont \color{black!70}
{\small Jack Sparrow \icon{\faEnvelopeO}{cvgreen}{} The Black Pearl \icon{\faMapMarker}{cvgreen}{} Tortuga \icon{\faPhone}{cvgreen}{} 0099/333 5647380 \newline\icon{\faAt}{cvgreen}{} \protect\url{jack@sparrow.com}
}
\end{center}
\end{minipage}

\end{paracol}

\end{document}
